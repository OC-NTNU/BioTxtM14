% Please write one sentence per line! (easier for version control)

\documentclass[10pt, a4paper]{article}
\usepackage{lrec2006}
\usepackage{graphicx}
\usepackage{todonotes}

\title{Towards text mining in environmental science:\\
extraction of quantitative variables and their relations}

\name{Author1, Author2, Author3}

\address{ Affiliation1, Affiliation2, Affiliation3 \\
               Address1, Address2, Address3 \\
               author1@xxx.yy, author2@zzz.edu, author3@hhh.com\\}


\abstract{
\todo[inline]{Abstract} 
Abstract\\ 
\newline 
\Keywords{Text Mining, Literature-based Knowledge Discovery, Environmental Science, Climate Science, Corpus Annotation, Relation Extraction}}

\begin{document}

\maketitleabstract

%=============================================================================
\section{Introduction}
%=============================================================================

Argue that text mining in environmental sciences:
has not been really pursued so far;
is different from text mining in biomedicine

Applications: search, QA, etc.

Explain literature-based knowledge discovery.

Related work (possibly in separate section or in Discussion section):
\cite{Hashimoto2012Excitatory}
\cite{Mihaila2013BioCause}
\cite{Zambach2010Lexical}
\cite{Vossen2008}


Goal and structure of curent paper.


%=============================================================================
\section{Annotation scheme}
%=============================================================================

Entities:Variables

Events: Change, Increase, Decrease, Cause, Correlate , Feedback

Other elements: combination (and/or), referring expressions

Attributes: Negation

%=============================================================================
\section{Rule extraction}
%=============================================================================

Proof that it is possible to fully automatically extract intended rule from proposed annotation. 

%=============================================================================
\section{Discussion}
%=============================================================================

Automatic annotation:
What aproach do we intend to use?
Generic, unsupervised open IE vs dedicated supervised system?
How well do we think this is going to work?
Bootstrapping and active learning to save annotation costs?

Variables:
Interpretation of variables in context (anaphora resolution and resolving referring expression);
Linking to ontologies

Reasoning:
How do want to use the extracted rules for reasoning?
Specification/generalization of rules (relation to natural logic?)
How to combine with bckground knowledge?

Context modelling:
Are there examples where context is essential for proper rule application (conditions?)

User interface:
What kind of use cases do we imagine?
What functionalities (e.g. clicking through to see source text of rule in publication )
What kind of user interface would support this? 

%=============================================================================
\section{Conclusion}
%=============================================================================


%=============================================================================
\section{Acknowledgements}
%=============================================================================
Financial aid from the European Commission (OCEAN-CERTAIN, FP7-ENV-2013-6.1-1; no: 603773) is gratefully acknowledged. 


\section{References}

\bibliographystyle{lrec2006}
\bibliography{biotxtm14}

\end{document}

