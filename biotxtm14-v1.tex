% Please write one sentence per line! (easier for version control)

\documentclass[10pt, a4paper]{article}
\usepackage{lrec2006}
\usepackage{graphicx}
\usepackage{todonotes}

\title{Towards text mining in environmental science:\\
extraction of quantitative variables and their relations}

\name{Author1, Author2, Author3}

\address{ Affiliation1, Affiliation2, Affiliation3 \\
               Address1, Address2, Address3 \\
               author1@xxx.yy, author2@zzz.edu, author3@hhh.com\\}


\abstract{
\todo[inline]{Abstract} 
Abstract\\ 
\newline 
\Keywords{Text Mining, Literature-based Knowledge Discovery, Environmental Science, Climate Science, Corpus Annotation, Relation Extraction}}

\begin{document}

\maketitleabstract

%=============================================================================
\section{Introduction}
%=============================================================================

Argue that text mining in environmental sciences:
has not been really pursued so far;
is different from text mining in biomedicine

Applications: search, QA, etc.

The scientific literature is growing so rapidly that researchers have been forced to become increasingly specialized to keep up with the state-of-the-art. 
This specialization can lead to the fragmentation of science, as researchers from different (sub-)disciplines rarely have time to read each other's papers. 
\newcite{Swanson1986Undiscovered} claimed that this fragmentation of science gives rise to \emph{undiscovered public knowledge}, i.e. still unmade inferences that can be made based on publicly available knowledge. As an example, he hypothesized that fish oils can cure Raynaud's disease by combining two publicly available statements: 1) Fish oils reduce blood viscosity, 2) patients with Raynaud's disease tend to exhibit high blood viscosity \cite{Swanson1986Fishoil}. 
This method of making hypothesis inferences of the form $A \to C$ based on publicly available statements $A \to B$ and $B \to C$ has been termed \emph{Swanson linking}\footnote{In the Swanson linking paradigm, the operator $\to$ is not interpreted as a logical conditional, but rather as an abstract causal or correlative relation between the terms. The conclusion $A \to C$ is therefore not logically sound, but has nevertheless been showed to give empirically useful results.}. 

The discoveries of Swanson prompted a line of research into methods for automatic identification of undiscovered public knowledge, an area which has been labelled Literature-based discovery (LBD). 
Most LBD methods are based on Swanson linking, and use term co-occurrence frequencies to detect relations of the type $A \to B$. 
Terms are usually extracted as n-grams from the text \cite{Lindsay1999LBDLexicalStat} or taken from a controlled vocabulary or ontology \cite{Weeber2001ConceptsInLBD}.

Recently, co-occurrence based methods have come under critique for leading to too many spurious discoveries.
\newcite{Hristovski2008NLPinLBD} therefore advocate an approach based on Natural Language Processing (NLP), as such approaches are able to be more specific about the relation that holds between two terms. 



Related work (possibly in separate section or in Discussion section):
\cite{Hashimoto2012Excitatory}
\cite{Mihaila2013BioCause}
\cite{Zambach2010Lexical}
\cite{Vossen2008}


Goal and structure of curent paper.


%=============================================================================
\section{Annotation scheme}
%=============================================================================

Entities:Variables

Events: Change, Increase, Decrease, Cause, Correlate , Feedback

Other elements: combination (and/or), referring expressions

Attributes: Negation

%=============================================================================
\section{Rule extraction}
%=============================================================================

Proof that it is possible to fully automatically extract intended rule from proposed annotation. 

%=============================================================================
\section{Discussion}
%=============================================================================

Automatic annotation:
What aproach do we intend to use?
Generic, unsupervised open IE vs dedicated supervised system?
How well do we think this is going to work?
Bootstrapping and active learning to save annotation costs?

Variables:
Interpretation of variables in context (anaphora resolution and resolving referring expression);
Linking to ontologies

Reasoning:
How do want to use the extracted rules for reasoning?
Specification/generalization of rules (relation to natural logic?)
How to combine with bckground knowledge?

Context modelling:
Are there examples where context is essential for proper rule application (conditions?)

User interface:
What kind of use cases do we imagine?
What functionalities (e.g. clicking through to see source text of rule in publication )
What kind of user interface would support this? 

%=============================================================================
\section{Conclusion}
%=============================================================================


%=============================================================================
\section{Acknowledgements}
%=============================================================================
Financial aid from the European Commission (OCEAN-CERTAIN, FP7-ENV-2013-6.1-1; no: 603773) is gratefully acknowledged. 


\section{References}

\bibliographystyle{lrec2006}
\bibliography{biotxtm14}

\end{document}

